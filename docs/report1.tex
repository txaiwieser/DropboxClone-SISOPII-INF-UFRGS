\documentclass[a4paper]{article}

%% Language and font encodings
\usepackage[brazilian]{babel}
\usepackage[utf8x]{inputenc}
\usepackage[T1]{fontenc}

%% Sets page size and margins
\usepackage[a4paper,top=3cm,bottom=2cm,left=3cm,right=3cm,marginparwidth=1.75cm]{geometry}

%% Useful packages
\usepackage{amsmath}
\usepackage{graphicx}
\usepackage[colorinlistoftodos]{todonotes}
\usepackage[colorlinks=true, allcolors=blue]{hyperref}

\title{Dropbox - Parte I}
\author{Augusto Bennemann, Fabrício Martins Mazzola e Txai Wieser}
\date{}
\begin{document}
\maketitle

%\begin{abstract}
%Your abstract.
%\end{abstract}

\section{Ambiente de desenvolvimento e teste}

O projeto foi desenvolvido utilizando um ambiente com as seguintes caracteristicas:
\begin{itemize}
	\item Configuração da máquina:
	\begin{itemize}
		\item Processador(es): Intel Core i5-2520M
		\item Memória: 2x Kingston 4GB 1333Mhz DDR3
	\end{itemize}

	\item{Sistema operacional}: 
	\begin{itemize}
		\item Distribuição: Debian GNU/Linux 
		\item Versão: Stretch (testing) Kernel 4.9.0-3-amd64
	\end{itemize}

	\item Software suporte:
	\begin{itemize}
		\item Compiladores (versões): gcc 6.3.0 20170516
	\end{itemize}
\end{itemize}

\section{Descrições e justificativas}

\subsection{Funcionamento}

A aplicação do cliente é composta por três threads. A primeira thread representa a interface entre os comandos disponíveis para o usuário e a comunicação entre cliente e servidor (shell). Ela é responsável por esperar comandos inseridos pelo usuário através da CLI e invocar as primitivas correspondentes na aplicação cliente. 

A segunda thread é um \textit{daemon} responsável por aguardar notificações do sistema de arquivos, através da biblioteca \textit{inotify}. O objetivo da thread é de monitorar as modificações realizadas nos arquivos presentes no diretório de sincronização do usuário (\textit{sync\_dir}). Quando um evento de criação ou renomeação de arquivo é constatado, essa thread automaticamente faz o upload desse arquivo para o servidor.

A terceira thread representa um servidor local (\textit{local\_server}), encarregado de ouvir chamadas de \textit{PUSH} e \textit{DELETE} vindas do servidor. Essas chamadas são utilizadas com o objetivo de manter a consistência e a sincronização entre os dispositivos dos clientes. Ao receber uma chamada \textit{PUSH filename}, a aplicação cliente realiza, de forma implícita, o download do arquivo \textit{filename}. De modo similar, o recebimento de uma chamada \textit{DELETE filename} fará a aplicação cliente remover o arquivo \textit{filename} da pasta de sincronização do usuário.

Ademais, foi criada uma lista de arquivos ignorados, utilizada para evitar que arquivos recém baixados ou removidos por ordem do servidor sejam capturados pelo \textit{daemon}. Quando a aplicação cliente recebe um \textit{PUSH} ou um \textit{DELETE} do servidor e, respectivamente, baixa ou remove o arquivo solicitado, o arquivo é inserido nessa lista. Assim, quando o \textit{inofity} percebe que algum desses arquivos é alterado, o encontra na lista de arquivos ignorados e então não propaga essas alterações ao servidor. Desse modo, evita-se situações em que diferentes dispositivos do mesmo usuário ficam enviando o mesmo arquivo um para o outro infinitamente. Por fim, o arquivo é excluído dessa lista, para que novas modificações nele possam ser capturadas pelo \textit{inotify}.

A aplicação servidor é composta por uma thread principal responsável por lidar com as requisições de conexões dos clientes. A cada tentativa de conexão que é aceita pelo servidor, uma nova thread é criada para o usuário solicitante. Essa thread aguarda requisições do cliente e gerencia o envio e remoção de arquivos na pasta \textit{server\_sync\_dir} do cliente. Além disso, quando um arquivo é recebido ou removido pelo servidor, a thread é encarregada de propagar essas alterações, através do envio de mensagens \textit{PUSH} e \textit{DELETE}, aos outros dispositivos conectados do usuário.

\subsection{Sincronização}

A sincronização da aplicação cliente foi feita utilizando-se duas variáveis de mutex: \textit{fileOperationMutex} e \textit{inotifyMutex}. A primeira considera uma operação sobre um arquivo como seção crítica e é usada para garantir que somente uma operação (p. ex. download, upload) seja realizada por vez. Já a segunda considera manipulações na lista de arquivos ignorados e nos \textit{timestamps} dos arquivos como seção crítica, de modo a garantir que o \textit{inotify} não seja ativado antes que um arquivo seja incluído na lista de arquivos ignorados e/ou tenha sua data de modificação original salva.

Na aplicação servidor, utilizou-se somente uma variável de mutex: \textit{clientCreationLock}. Essa variável é armazenada na struct de cada usuário e serve para garantir que a conexão de clientes ao servidor sejam executada em somente um dispositivo por vez.

Além de proteção de seções críticas, duas barreiras - \textit{syncbarrier} e \textit{localserverbarrier} - foram utilizadas para realizar a sincronização entre diferentes tarefas do cliente. A primeira barreira garante que o \textit{shell} somente seja exibido para o usuário depois que a conexão com o servidor for realizada com sucesso. Enquanto a thread principal permanece bloqueada esperando a sincronização ser concluída, uma mensagem de "Syncing..." é exibida ao usuário. Uma vez terminada a sincronização, uma mensagem de "Done" é exibida e então o programa passa a esperar pelos comandos do usuário.
A segunda barreira garante que a sincronização inicial de arquivos só será feita após o servidor local (\textit{local\_server}) estar disponível, ouvindo requisições do servidor do Dropbox.

\subsection{Comunicação}
A comunicação entre os clientes e o servidor é realizada por meio de sockets TCP. No momento em que a conexão TCP é estabelecida, um socket único é criado para cada dispositivo de cliente. Os comandos básicos da comunicação são descritos abaixo:


Cliente -> Servidor:

\begin{itemize}
\item LIST:
\newline Cliente pede ao servidor uma lista com os arquivos salvos. O servidor responde confirmando a requisição, seguido pela quantidade de arquivos e pelos nomes dos mesmos.

\item DOWNLOAD 'file':
\newline Cliente pede ao servidor o arquivo definido em file. O servidor responde confirmando a requisição, seguido pelo tamanho do arquivo e os dados do arquivo em si.

\item UPLOAD 'file':
\newline Cliente comunica ao servidor que um arquivo deseja ser enviado seguido pela sua data de modificação. O servidor compara se o arquivo do cliente é mais recente do que o do servidor. Nesse caso, o servidor responde aceitando a transação, e assim que o cliente recebe a confirmação do request, envia o tamanho do arquivo seguido pelos dados do arquivo em si.

\item DELETE 'file'
\newline Cliente avisa ao servidor que um arquivo deve ser deletado.

\item SYNC:
\newline Cliente pede para ao servidor para realizar a sincronia do diretório. Nesse caso o servidor responde o numero de arquivos a serem sincronizados. E para cada arquivo a ser sincronizado, o servidor começa o envio através da comunicação \textit{PUSH 'file'}.
\end{itemize}

Servidor -> Cliente:

\begin{itemize}

\item PUSH 'file':
\newline Servidor avisa ao cliente que um arquivo deve ser baixado, a partir dessa mensagem o cliente envia um request \textit{DOWNLOAD 'file'}.

\item DELETE 'file' [server -> client]:
\newline Servidor avisa ao cliente que um arquivo deve ser deletado.

\end{itemize}

Ademais, o mecanismo de sinais também foi utilizado. Quando o processo servidor é encerrado por meio da combinação de teclas \textit{CTRL + c}, o sinal \textit{SIGINT} é captado. O servidor, então, encerra sua execução de modo gracioso, fechando as conexões abertas. Isso permite que os clientes conectados instantaneamente percebam que a conexão foi fechada e, então, também encerrem sua execução com uma mensagem alertando que o servidor desconectou.

\subsection{Estruturas e funções adicionais}

\subsubsection{dropboxUtil}
\begin{itemize}
	\item void makedir\_if\_not\_exists(const char* path)
	\newline Verifica se o diretório especificado no parâmetro \textit{path} existe e caso contrario o cria.
	
	\item int file\_exists(const char* path)
	\newline Verifica se o arquivo especificado no parâmetro \textit{path} existe.

	\item int connect\_server(char* host, int port)
	\newline Abre uma conexão via socket com o \textit{host}.
\end{itemize}


\subsubsection{dropboxServer}
\begin{itemize}
	\item struct file\_info
	\newline Foram feitas duas mudanças na estrutura \textit{client}. A primeira foi a remoção da propriedade \textit{char extension[MAXNAME]}, pois escolhemos armazenar o nome completo do arquivo em \textit{char name[MAXNAME]} e, então, não havia motivos para manter essa informação duplicada. A outra modificação foi o uso de uma estrutura 'time\_t' ao invés de um 'char *' para a propriedade 'last\_modified'.
	
	\item struct client
	\newline Na estrutura \textit{client} foram adicionadas duas propriedades. A primeira é a \textit{int devices\_server[MAXDEVICES]} que armazena o socket \textit{servidor} do cliente (que recebe requisições de PUSH e DELETE). Além disso, foi adicionado o \textit{pthread\_mutex\_t mutex}, que é o mutex utilizado para proteger seções críticas de cada cliente, para evitar que sejam executadas simultaneamente em diferentes dispositivos.
	
	\item struct tailq\_entry
	\newline Estrutura usada na lista de clientes.
	
	\item void delete\_file(char *file)
	\newline Remove o arquivo \textit{file} do servidor. 'file' – filename.ext
	
	\item void *connection\_handler(void *socket\_desc)
	\newline Função executada pela thread que trata as conexões de cada cliente.

\end{itemize}


\subsubsection{dropboxClient}
\begin{itemize}
	\item struct tailq\_entry
	\newline Estrutura usada na lista de clientes.
	
	\item void delete\_server\_file(char *file)
	\newline Remove o arquivo file para do servidor. 'file' – filename.ext
\end{itemize}

\subsection{Testes}
Para testar a aplicação cliente e servidor, utilizamos métodos não automatizados (manuais). Foram realizados testes das funções da especificação através da interface de linha de comando (CLI) da aplicação cliente e os resultados foram sendo verificados nos diretórios do servidor e de clientes.

Um ponto importante e difícil de testar na prática foram as proteções de seções críticas. Para isso, tentou-se simular as condições de corrida de forma a realizar testes antes e depois da adição das variáreis de \textit{mutex} com o objetivo de comprovar a eficácia das mesmas.

\section{Dificuldades encontradas}

\subsection{Data de modificação incorreta}
Ao receber arquivos, o servidor às vezes os gravava com a data de modificação incorreta. A origem do problema estava no cliente, que transmitia a data errada. Isso estava ocorrendo pois quando um arquivo no diretório de sincronização do usuário era alterado, o daemon prontamente capturava a alteração, muitas vezes antes que a data de modificação fosse definida para a data do arquivo original. Assim, a solução encontrada foi utilizar um \textit{sleep} para garantir que o daemon espere alguns milissegundos antes de tratar o evento.

\subsection{Leitura de inteiros errados pelo socket}
Ao transmitir dados numéricos pelo socket, algumas vezes o valor chegava aparentemente corrompido. A saber, diversas vezes esse número era o tamanho do arquivo que seria lido em seguida, o que levava o receptor a ler menos caracteres que o esperado ou a ficar trancado esperando por caracteres que nunca chegariam. O problema estava na leitura feita anteriormente, que lia mais caracteres do que de fato a string enviada possuía. A solução encontrada, portanto, foi definir um tamanho fixo para essas mensagens, de modo que a leitura não recebesse bytes do dado numérico a ser lido em seguida.

\subsection{Utilização de exclusões mútuas}
A princípio, definir exclusões mútuas parecia simples, mas logo descobrimos que essa tarefa pode ser bastante complexa, pois exige pensar a fundo o comportamento de todas as threads simultaneamente, em suas diferentes possibilidades de intercalação. Foram várias as situações em que o programa chegava a uma situação de \textit{deadlock}.
 
\subsection{Sincronização de alterações \textit{offline}}
Seria ótimo que as alterações feitas durante o período em que o usuário esteve \textit{offline} pudessem ser refletidas no servidor assim que ele se conectasse. Na tentativa de implementar essa funcionalidade, fizemos com que o cliente automaticamente salve no seu diretório um arquivo \textit{.dropboxfiles} contendo nome e data de modificação de cada um de seus arquivos. 

Assim, na próxima vez que ele se conectasse seria possível descobrir, comparando esse arquivo e os arquivos no diretório, quais alterações foram feitas no período de tempo que o usuário esteve desconectado. 
No entanto, apesar de termos começado a implementação, não a concluímos por falta de tempo.

\section{Conclusões}
Ao terminar a implementação do trabalho, percebemos o quanto aprendemos com a construção do mesmo. Ao analisar a especificação vimos que teríamos que aprender diversas técnicas para construir a aplicação, como notificações do sistema de arquivos, comunicação utilizando sockets, proteções de seções críticas, barreiras, threads, e tudo isso utilizando uma linguagem de mais baixo nível. Mas construindo aos poucos, fomos estruturando nosso conhecimento e cada vez entendendo mais sobre o escopo envolvido.

A oportunidade de poder colocar em prática os conhecimentos adquiridos na disciplina (e no restante do curso) em um trabalho com conceitos e funções bem atuais foi fantástica. Implementar em um trabalho funções similares a serviços/produtos populares (como o Dropbox) nos faz relacionar e visualizar os conceitos aprendidos com o que realmente utilizamos no nosso dia-a-dia.

\end{document}