\documentclass[a4paper]{article}

%% Language and font encodings
\usepackage[brazilian]{babel}
\usepackage[utf8x]{inputenc}
\usepackage[T1]{fontenc}

%% Sets page size and margins
\usepackage[a4paper,top=3cm,bottom=2cm,left=3cm,right=3cm,marginparwidth=1.75cm]{geometry}

%% Useful packages
\usepackage{amsmath}
\usepackage{graphicx}
\usepackage[colorinlistoftodos]{todonotes}
\usepackage[colorlinks=true, allcolors=blue]{hyperref}

\title{Dropbox - Parte I}
\author{Augusto Bennemann, Fabrício Martins Mazzola e Txai Wieser}

\begin{document}
\maketitle

\begin{abstract}
Your abstract.
\end{abstract}

\section{Ambiente de desenvolvimento e teste}

O projeto foi desenvolvido utilizando um ambiente com as seguintes caracteristicas:
\begin{itemize}
	\item Configuração da máquina:
	\begin{itemize}
		\item Processador(es):
		\item Memória:
	\end{itemize}

	\item{Sistema operacional}: 
	\begin{itemize}
		\item Distribuição:
		\item Versão:
	\end{itemize}

	\item Software suporte:
	\begin{itemize}
		\item Compiladores (versões):
	\end{itemize}
\end{itemize}

\section{Descrições e justificativas}

\subsection{Funcionamento}
(threads, metodos do ‘protocolo de aplicação’)

\subsection{Concorrência no servidor}

\subsection{Sincronização}

\subsection{Comunicação}

\subsection{Estruturas e funções adicionais}

\subsubsection{dropboxUtil}
\begin{itemize}
	\item void makedir\_if\_not\_exists(const char* path)
	\newline Verifica se o diretório especificado no parâmetro `path` existe e caso contrario o cria.
	
	\item int file\_exists(const char* path)
	\newline Verifica se o diretório especificado no parâmetro `path` existe.

	\item int connect\_server(char * host, int port)
	\newline Abre uma conexão via socket com o `host`.
\end{itemize}


\subsubsection{dropboxServer}
\begin{itemize}
	\item struct file\_info
	\newline Foram feitas duas mudanças na estrutura 'client', a primeira foi a remoção da propriedade 'char extension[MAXNAME]' para evitar duplicações ja que a extensão ja é facilmente recuperada através de 'char name[MAXNAME]'. A outra modificação foi o uso de uma estrutura 'time\_t' ao invés de um 'char *' para a propriedade 'last\_modified'.
	
	\item struct client
	\newline Na estrutura 'client' foram adicionadas duas propriedades, a primeira é a 'int devices\_server[MAXDEVICES]' que armazena sockets 'servidor' do cliente (que recebe requisições de PUSH e DELETE), 'pthread\_mutex\_t mutex' foi adicionado e é o mutex que protege as seções críticas de cada cliente.
	
	\item struct tailq\_entry
	\newline Estrutura usada na lista de clientes.
	
	\item void delete\_file(char *file)
	\newline Remove o arquivo file para do servidor. 'file' – filename.ext
	
	\item void *connection\_handler(void *socket\_desc)
	\newline Trata as conexões para cada cliente.

\end{itemize}


\subsubsection{dropboxClient}
\begin{itemize}
	\item struct tailq\_entry
	\newline Estrutura usada na lista de clientes.
	
	\item void delete\_server\_file(char *file)
	\newline Remove o arquivo file para do servidor. 'file' – filename.ext
\end{itemize}
\subsection{Testes}

\section{Dificuldades encontradas}
 
\bibliographystyle{alpha}
\bibliography{sample}

\end{document}